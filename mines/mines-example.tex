\documentclass[aspectratio=169]{beamer}
\usepackage[utf8]{inputenc}
\usepackage[T1]{fontenc}

\usetheme{mines}
% \usefonttheme{serif} % comment for san-serif font

% Put the Table of Contents at the beginning
% of each section and highlight the title
% of the current section
\AtBeginSection[]
{
  \begin{frame}
    \frametitle{Table of Contents}
    \tableofcontents[currentsection]
  \end{frame}
}

% AMS and mathtools
\usepackage{amsmath,amsthm,amssymb,marvosym,mathrsfs,amsfonts,amscd,mathtools}

% Hyperlinks and URLs
\usepackage{url}
\usepackage{hyperref}
\hypersetup{
    colorlinks,
    citecolor=black,
    filecolor=black,
    linkcolor=black,
    urlcolor=black
}

% Colors
\usepackage{xcolor}

% Bold math
\usepackage{bm}

% Bra Ket (Dirac) Notation
\usepackage{braket}

% Slashed characters (e.g. in Dirac equation)
\usepackage{slashed}

% Clean SI Units
\usepackage{siunitx}

% Enumerate thingies
\usepackage{enumitem}

% Cancel things out in equations
\usepackage[makeroom]{cancel}

% Graphics and figures
\usepackage{graphicx}
\usepackage{wrapfig}
\usepackage{float}

% Caption figures and tables
\usepackage{caption}

% Generate symbols
\usepackage{textcomp} % Include this line to avoid output errors
\usepackage{gensymb}

% Make multiple rows in a table
\usepackage{multirow}

% Booktabs tables
\usepackage{booktabs}

% Useful frames
\usepackage{mdframed}

% Comment-out large sections
\usepackage{comment}

% No auto-indent
\setlength{\parindent}{0pt}

% Asymptote - 3D vector graphics
\usepackage{asymptote}

% Tikz Package Stuff
\usepackage{pgf,tikz,pgfplots}
\usepackage{tikz-3dplot}
% Use various tikz libraries
\usetikzlibrary{decorations.pathmorphing, decorations.markings, decorations.pathreplacing, patterns} % Decorate paths!
\usetikzlibrary{calc}
\usetikzlibrary{scopes}
\usetikzlibrary{angles, quotes}
\usetikzlibrary{svg.path}
\usetikzlibrary{arrows, arrows.meta}
\usetikzlibrary{fadings}
% pgfplots package settings
\pgfplotsset{compat=1.15}
% \pgfplotsset{width=10cm,compat=1.9} % Taken from latest overleaf.


% Awesome circled numbers
\newcommand*\circled[4]{\tikz[baseline=(char.base)]{\node[shape=circle, fill=#2, draw=#3, text=#4, inner sep=2pt] (char) {#1};}}

% Control size of text
\usepackage{relsize}

% Extend conditional commands
\usepackage{xifthen}

% Scale math by size
\newcommand*{\Scale}[2][4]{\scalebox{#1}{\ensuremath{#2}}}

% Big integrals
\usepackage{bigints}

% Number equations within sections
\numberwithin{equation}{section}

% Generate blind text
\usepackage{blindtext}

% Useful symbols
\usepackage{marvosym}


%%%% BLACKBOARD BOLD %%%%
\newcommand{\bbN}{\mathbb{N}} % Natural numbers
\newcommand{\bbZ}{\mathbb{Z}} % Zahlen
\newcommand{\bbQ}{\mathbb{Q}} % Rational numbers
\newcommand{\bbR}{\mathbb{R}} % Real numbers
\newcommand{\bbC}{\mathbb{C}} % Complex numbers
\DeclareSymbolFont{bbold}{U}{bbold}{m}{n} % Identity matrix
\DeclareSymbolFontAlphabet{\mathbbold}{bbold} % Identity matrix
\newcommand{\identitymatrix}{\mathbbold{1}} % Identity matrix


%%%% CODE LISTING %%%%
\usepackage{listings}
\definecolor{greencomments}{HTML}{00BA00}
\definecolor{graynumbers}{HTML}{4F4F4F}
\definecolor{purplestrings}{HTML}{AD00AA}
\definecolor{backgroundcolor}{HTML}{E8E8E8}
\lstdefinestyle{nkostin}{
    backgroundcolor=\color{backgroundcolor},   
    commentstyle=\color{greencomments},
    keywordstyle=\color{blue},
    numberstyle=\tiny\color{graynumbers},
    stringstyle=\color{purplestrings},
    basicstyle=\footnotesize,
    breakatwhitespace=false,         
    breaklines=true,                 
    captionpos=b,                    
    keepspaces=true,                 
    numbers=left,                    
    numbersep=5pt,                  
    showspaces=false,                
    showstringspaces=false,
    showtabs=false,                  
    tabsize=2
}
\lstset{style=nkostin}

%%%% UNIT BASIS VECTORS %%%%
\newcommand{\ihat}{\bm{\hat{\imath}}} % Cartesian i hat (x-direction)
\newcommand{\jhat}{\bm{\hat{\jmath}}} % Cartesian j hat (y-direction)
\newcommand{\khat}{\bm{\hat{k}}} % Cartesian k hat (z-direction)
\newcommand{\rhat}{\bm{\hat{r}}} % Spherical r hat
\newcommand{\phihat}{\bm{\hat{\phi}}} % Spherical phi hat
\newcommand{\thetahat}{\bm{\hat{\theta}}} % Spherical theta hat
\newcommand{\nhat}{\bm{\hat{n}}} % Unit normal vector
\newcommand{\rhohat}{\bm{\hat{\rho}}} % Cylindrical rho hat
\newcommand{\zhat}{\bm{\hat{z}}} % Cylindrical z hat


%%%% COLORS: DEFINITIONS AND COMMANDS %%%%


% Number equations within sections
\numberwithin{equation}{section}

% Continuous per-section numbering of figures and tables
\usepackage{chngcntr}
\counterwithin{figure}{section}
\counterwithin{table}{section}

\title{Brief Introduction to the Beamer Class}
\subtitle{A Primer on Making Awesome Presentations}

\author{Nicholas D. Kostin}
\date{04 April 2022}

\begin{document}

\frame{\titlepage}

\begin{frame}
    \frametitle{Table of Contents}
    \tableofcontents
\end{frame}

\section{First Section}

\begin{frame}
\frametitle{First Frame Title}
\framesubtitle{Subtitle pertaining to the first frame.}

Frame contents.

\end{frame}

\begin{frame}
\frametitle{Second Frame Title}
\framesubtitle{Subtitle pertaining to the second frame.}

Some important text will be \alert{highlighted} because it's important.

\vfill

\begin{block}{Block Box}
    The text contained inside this block serves some purpose.
\end{block}

\end{frame}

\section{Second Section}

\begin{frame}
\frametitle{Third Frame Title}
\framesubtitle{Subtitle pertaining to the third frame.}

This frame contains a figure.

\begin{figure}[H]
\centering
\captionsetup{width=0.8\textwidth,labelfont={color=black,bf},textfont={color=black}}
    \includegraphics[height=0.5\textheight]{../img/funny.png}
\caption{A funny picture.}
\end{figure}

\end{frame}

\begin{frame}
\frametitle{Fourth Frame Title}
\framesubtitle{Subtitle pertaining to the fourth frame.}

\begin{columns}

\column{0.5\textwidth}

The Dirac delta can be loosely thought of a function on the real line which is zero everywhere except at the origin, where it is infinite:
\begin{equation}
    \delta (x) = \begin{cases} +\infty, &x = 0 \\ 0, &x \neq 0, \end{cases}
\end{equation}

and which is also constrained to satisfy the identity
\begin{equation}
    \int\limits_{\bbR} \delta (x)\ dx = 1.
\end{equation}

\column{0.5\textwidth}

\begin{figure}[H]
\centering
\captionsetup{width=0.8\textwidth,labelfont={color=black,bf},textfont={color=black}}
    \includegraphics[height=0.6\textheight]{../img/cute.jpg}
\caption{A cute picture.}
\end{figure}

\end{columns}

\end{frame}

\begin{frame}
\frametitle{Fifth Frame Title}
\framesubtitle{Subtitle pertaining to the fifth frame.}

\begin{columns}

\column{0.5\textwidth}
Here is some text in the first column. A famous equation is written below.

\begin{equation*}
    \frac{1}{c^2} \frac{\partial^2}{\partial t^2} \psi - \nabla^2 \psi + \frac{m^2 c^2}{\hbar^2} \psi = 0.
\end{equation*}

\column{0.5\textwidth}
This text will be in the second column.

\begin{itemize}
    \item[$\bullet$] First list item
    \item[$\bullet$] Second list item
    \item[$\bullet$] Third list item
\end{itemize}

\end{columns}

\end{frame}

\begin{frame}
\frametitle{Sixth Frame Title}
\framesubtitle{Subtitle pertaining to the sixth frame.}

Consider some $A \in \bbR^{p \times q}$ and $B \in \bbR^{q \times p}$. We can represent these matrices as
\begin{equation*}
    A = \begin{pmatrix} a_{11} & a_{12} & \cdots & a_{1q} \\ a_{21} & a_{22} & \cdots & a_{2q} \\ \vdots & \vdots & \ddots & \vdots \\ a_{p1} & a_{p2} & \cdots & a_{pq} \end{pmatrix} \qquad \text{and} \qquad B = \begin{pmatrix} b_{11} & b_{12} & \cdots & b_{1p} \\ b_{21} & a_{22} & \cdots & b_{2p} \\ \vdots & \vdots & \ddots & \vdots \\ b_{q1} & b_{q2} & \cdots & b_{qp} \end{pmatrix}
\end{equation*}

respectively. Then the product $AB$ can be written
\begin{equation*}
    AB = \begin{pmatrix} a_{11} b_{11} + \cdots + a_{1q} b_{q1} & a_{11} b_{12} + \cdots + a_{1q} b_{q2} & \cdots & a_{11} b_{1p} + \cdots + a_{1q} b_{qp} \\ a_{21} b_{11} + \cdots + a_{2q} b_{q1} & a_{21} b_{12} + \cdots + a_{2q} b_{q2} & \cdots & a_{21} b_{1p} + \cdots + a_{2q} b_{qp} \\ \vdots & \vdots & \ddots & \vdots \\ a_{p1} b_{11} + \cdots + a_{pq} b_{q1} & a_{p1} b_{12} + \cdots + a_{pq} b_{q2} & \cdots & a_{p1} b_{1p} + \cdots + a_{pq} b_{qp} \end{pmatrix}.
\end{equation*}

\end{frame}

\end{document}
